\chapter{Relativistic field theory}

Quantum field theory (QFT) is the quantum extention of relativistic field theory.

\section{The Schrödinger equation}

Classical mechanics describes the time evolution of objects with definite spatial coordinates.
Quantum mechanics extends this by describing the state of a system with some wavefunction $\psi$
which is acted upon by operators. This admits two kinds of descriptions of a quantum system: one in
the language of differential equations where the operators are various combinations of
multiplication and differentiation, and another in the language of linear algebra where the quantum
states are vectors and the operators are matrices. To many students in chemistry, especially those
without a background in linear algebra, this can be confusing.

The unification of these perspectives comes from the fact that functions can be used to construct a
vector space. The only requirement for elements of a vector space is that they may be added to each
other and scaled by some constant. In the general case of quantum mechanics, the space of all
square-integrable functions

\begin{equation}
i \hbar \frac{\partial}{\partial t} \psi = \hat{H} \psi
\end{equation}
The Hamiltonian $\hat{H}$ is an operator that determines the time evolution of the quantum state
$\psi$, and in the simple cases usually considered in introductory quantum mechanics, it is
separable into a linear combination of a position-dependent portion (the potential energy) and a
momentum-dependent portion (the kinetic energy):
\begin{flalign}
\hat{H} & = \hat{T} + \hat{V}       \\
\hat{T} & = \frac{\hat{p}^2}{2m}    
\end{flalign}
We define the momentum operator $\hat{p}$ below, providing a complete expression for the kinetic
energy operator:
\begin{flalign}
\hat{p} & = -i \hbar \vec{\nabla}   \\
\hat{T} & = \frac{-\hbar^2}{2m} \vec{\nabla}
\end{flalign}

One common question that arises in introductory quantum mechanics is why the momentum operator takes
the form that it does. We know that higher frequency waves have a higher momentum (per the de 
Broglie hypothesis), and a momentum operator should exploit this property. The spatial derivative of
a plane wave $e^{ikx}$ with frequency $k$ happens to extract this factor:
\begin{equation*}
\frac{\partial}{\partial x} e^{ikx} = ik e^{ikx}
\end{equation*}
The only problem is that this eigenvalue equation is imaginary-valued, and we want our operators to
return real-valued eigenvalues. To eliminate the factor of $i$, we multiply by $-i$. The reduced
Planck constant $\hbar$ serves as a natural scale factor relating the momentum to the wavelength --
in some cases, it may be more convenient to work in a unit system where $\hbar = 1$.

Observable quantities are generally representable by real numbers

In many cases, we are interested in finding stationary states of the equation, which we can
accomplish with the \textit{time-independent Schrödinger equation}:
\begin{equation}
\hat{H} \psi = E \psi
\end{equation}
This is an eigenvalue equation: solving this equation results in eigenvectors -- the stationary
states, or \textit{eigenstates} -- whose eigenvalues are the energies associated with each state.
Crucially, these eigenstates can serve as a basis in many cases that can describe any other possible
state of the system. However, it is important to remember that \textit{quantum mechanics describes 
the evolution of states over time}, so the time-independent equation is not generally applicable to
any quantum system.

\section{The Klein-Gordon equation}

The Schrödinger equation suffers a serious problem. It does not treat space and time on equal
footing. This is clear when we look at the orders of derivatives involved in the time-dependent
Schrödinger equation: on the left side we have a first-order derivative in time, but on the right
we have a second-order spatial derivative. At the time of Erwin Schrödinger's derivation of his
equation in 1926, the mathematics of special relativity had already been developed, and he would
have been well aware of this issue.

Schrödinger derived his equation from another equation which did, in fact have the right properties
to ensure its relativistic invariance. As we will later see, Schrödinger had good reason to derive a
non-relativistic version of this equation outside of simplifying the math describing system where
relativity was unimportant. However, even if this equation is limited in its computational utility,
it is still critical to understanding quantum field theory.

We start with Einstein's famous energy equation -- not simply $E = mc^2$, but the full version that
includes momentum:
\begin{equation}
E^2 = p^2 c^2 + m^2 c^4
\end{equation}
Knowing how the Schrödinger equation is structured, it's tempting to perform canonical quantization
and think of this equation as operators acting on a wavefunction:
\begin{equation}
\hat{E}^2 \psi = \left(\hat{p}^2 c^2 + m^2 c^4\right) \psi
\end{equation}
As with the Schrödinger equation, $\psi$ can be a real or complex valued square-integrable function.
We can substitute the operator definitions into this equation: $\hat{p} = i\hbar\vec{\nabla}$, and
$\hat{E} = i\hbar\frac{\partial}{\partial t}$:
\begin{flalign}
\left(i\hbar\frac{\partial}{\partial t}\right)^2 \psi
    & = \left(c^2\left(i\hbar\vec{\nabla}\right)^2 + m^2 c^4\right) \psi    \\
-\hbar^2 \frac{\partial^2}{\partial t^2} \psi
    & = \left(-\hbar^2 c^2 \vec{\nabla}^2 + m^2 c^4\right) \psi
\end{flalign}
For the sake of simplicity, we'll divide everything by $\hbar^2 c^2$:
\begin{flalign}
-\frac{1}{c^2}\frac{\partial^2}{\partial t^2} \psi
    & = \left(-\vec{\nabla}^2 + \frac{m^2 c^2}{\hbar^2}\right) \psi
\end{flalign}
Now we'll group our differential operators on the left, which allows us to identify the geometric
derivative operator of the spacetime algebra, $\partial = \vec{\nabla} -
\frac{1}{c} \frac{\partial}{\partial t}$:
\begin{flalign}
\left(\vec{\nabla}^2 - \frac{1}{c^2}\frac{\partial^2}{\partial t^2}\right) \psi
    & = \frac{m^2 c^2}{\hbar^2} \psi    
\end{flalign}

This equation is the \textit{Klein-Gordon equation}. Sometimes, we will see it presented in a
slightly different way:
\begin{equation}
\left(\partial^2 - \frac{m^2 c^2}{\hbar^2}\right) \psi = 0
\end{equation}
It describes the dynamics of a scalar field in free space. However, it suffers a critical problem:
it is a second-order differential equation. Determining the field evolution requires two initial
conditions, and more importantly, the density $\rho = \psi^* \psi$ cannot be guaranteed to be
positive-definite throughout the field evolution. A probabilistic interpretation of the theory is
not possible, as the Born rule cannot be applied.

\subsection{Reduction to the Schrödinger equation}

In the Galilean limit of low velocity, the Klein-Gordon equation reduces to the Schrödinger
equation. Essentially, we provide one of the initial conditions to the Klein-Gordon equation.

\section{The Dirac equation}

Solving the issues with the Klein-Gordon equation while maintaining its relativistic invariance
proved difficult. The most likely route to a solution involved taking the square root of the
energy equation used previously:
\begin{equation}
E = \sqrt{p^2 c^2 + m^2 c^4}
\end{equation}
This is simple enough when we deal with energy and momentum as numerical quantities, but promoting
them to operators results in a serious problem: how can we take the square root of a differential
operator, especially in combination with a constant? The first options that came to mind were
to use a Taylor or Fourier expansion, but simplifying the likely infinite series that would arise
from 

Paul Dirac came up with a brilliant solution this problem -- but the origins of it almost feel like
fitting a round peg into a square hole. He treated the wave operator as if it were simply the square
of the sum of relativistic partial derivatives:
\begin{equation}
\left(\vec{\nabla}^2 - \frac{1}{c^2}\frac{\partial^2}{\partial t^2}\right) \psi = \left(
    \gamma_0 \frac{\partial}{\partial t} +
    \gamma_1 \frac{\partial}{\partial x} + 
    \gamma_2 \frac{\partial}{\partial y} + 
    \gamma_3 \frac{\partial}{\partial z} \right)^2 \psi
\end{equation}
The obvious issue with this naive expansion of the wave operator is the existence of cross terms,
such as $\frac{\partial}{\partial t} \frac{\partial}{\partial x}$. Dirac realized that in order to
eliminate the cross terms, some factors, shown here as $\gamma_\mu$, would have to be zero when
multiplied with each other. He realized that the factors would have to have the following
properties:
\begin{flalign}
\gamma_\mu \gamma_\nu & = -\gamma_\nu \gamma_\mu  \\
\gamma_0^2 & = -1                               \\
\left(\gamma_1\right)^2 = \left(\gamma_2\right)^2 = \left(\gamma_3\right)^2 & = 1
\end{flalign}
These conditions can be stated in terms of the anticommutator between $\gamma_\mu$ factors and the
Minkowski metric:
\begin{flalign}
\frac{1}{2}\left\{\gamma_\mu, \gamma_\nu\right\} = \eta_{\mu\nu}
\end{flalign}
Dirac realized that the $\gamma_\mu$ factors could not be ordinary numbers, but could be expressed
in terms of matrices. These matrices are known as the \textit{gamma matrices} or \textit{Dirac
matrices}, which have the canonical representation given below:
\begin{align*}
    \gamma_0 &= \begin{bmatrix}
         1 &  0 &  0 &  0 \\
         0 &  1 &  0 &  0 \\
         0 &  0 & -1 &  0 \\
         0 &  0 &  0 & -1 
    \end{bmatrix}, &
    \gamma_1 &= \begin{bmatrix}
         0 &  0 &  0 &  1 \\
         0 &  0 &  1 &  0 \\
         0 & -1 &  0 &  0 \\
        -1 &  0 &  0 &  0 
    \end{bmatrix}, \\
    \\
    \gamma_2 &= \begin{bmatrix}
         0 &  0 &  0 & -i \\
         0 &  0 &  i &  0 \\
         0 &  i &  0 &  0 \\
        -i &  0 &  0 &  0 
    \end{bmatrix}, &
    \gamma_3 &= \begin{bmatrix}
         0 &  0 &  1 &  0 \\
         0 &  0 &  0 & -1 \\
        -1 &  0 &  0 &  0 \\
         0 &  1 &  0 &  0 
    \end{bmatrix}
\end{align*}
