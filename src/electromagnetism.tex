\chapter{Electromagnetism}

\section{Differential operators}



\section{Maxwell's equations}

Those of you who took an undergraduate course in electromagnetism are likely familiar with Maxwell's
equations:
\begin{flalign}
    \vec{\nabla}  \cdot \vec{E} & = \frac{\rho}{\epsilon_0}                 \\
    \vec{\nabla}  \cdot \vec{B} & = 0                                       \\
    \vec{\nabla} \times \vec{E} & = -\frac{\partial \vec{B}}{\partial t}    \\
    \vec{\nabla} \times \vec{b} & = \mu_0 J + \frac{1}{c^2} \frac{\partial \vec{B}}{\partial t}
\end{flalign}

The first of these equations is \textit{Gauss's law}, which states that the divergence of the
electric field is determined by the charge density at that point. The second is \textit{Gauss's law
for magnetism}, which states that the magnetic field has zero divergence - and by implication, there
are no sources of magnetic charge (magnetic monopoles). The third is \textit{Faraday's law}, which
states that a time-varying magnetic field induces a curl in the electric field -- in other words,
the charge density does not completely determine the electric field; changing magnetic fields
contribute to it as well. The last equation is Ampere's law, which states that the current density
-- the motion of charge in space -- generates the electric field, with a contribution from the
change in electric field over time.

\section{The electromagnetic field in the algebra of physical space}

The conventional statement of Maxwell's equations is in the language of vector algebra. We'll spend
some time converting this to the language of geometric algebra.

The first and most obvious change we can make is replacing the cross product with a wedge product.
This will give us some insight into the graded structure of the electromagnetic field. The curl
operator, $\vec{\nabla} \times \vec{f}$, is conventionally thought of as analogous to a cross
product. We can define a new curl-like differential operator on vector fields that returns a
bivector field, analogous to the wedge product:
$\vec{\nabla} \wedge \vec{f} = i \left(\vec{\nabla} \times \vec{f}\right)$. We substitute these
definitions into Faraday's law and Ampere's law:
\begin{flalign}
    i \left(\vec{\nabla} \wedge \vec{E}\right) & = -\frac{\partial \vec{B}}{\partial t} \\
    i \left(\vec{\nabla} \wedge \vec{B}\right) & = \mu_0 J + \frac{\partial \vec{E}}{\partial t}
\end{flalign}

Next, we'll make a change in our description of the magnetic field. Recall the Lorentz force law:
\begin{equation}
\vec{f_{\text{EM}}} = q (\vec{E} + \vec{v} \times \vec{B})
\end{equation}
The cross product suggests that $\vec{B}$ may not truly be a vector. Of $\vec{f_{\text{EM}}}$,
$\vec{v}$, and $\vec{B}$, one must be a pseudovector. The physical link between magnetic fields and
the rotation of charges encourages us to rephrase the Lorentz force law with a wedge product. We
also alter our description of the magnetic field so that it units are commensurate with those of the
electric field:
\begin{flalign}
\vec{f_{\text{EM}}}
    & = q (\vec{E} + \vec{v} \wedge i\vec{B})   \\
    & = q (\vec{E} + \frac{\vec{v}}{c} \wedge ic\vec{B})
\end{flalign}
We can now define the \textit{Riemann-Silberstein multivector} $F = \vec{E} + ic\vec{B}$ from the
combination of the electric and magnetic fields.

Now that we've made these changes to the magnetic field, we can propagate them to Faraday's law and
Ampere's law. We'll ensure that all instances of $\vec{B}$ become $ic\vec{B}$:
\begin{flalign}
\vec{\nabla} \wedge \vec{E} & = \frac{\partial}{\partial t} ic\vec{B}         \\
\vec{\nabla} \wedge ic\vec{B} & = \mu_0 J + \frac{\partial \vec{E}}{\partial t}
\end{flalign}
Now we can add the two laws together to get a combined Faraday-Ampere law that acts on the
Riemann-Silberstein multivector:
\begin{flalign}
\vec{\nabla} \wedge \left(\vec{E} + ic\vec{B}\right)
    & = \frac{\partial}{\partial t} ic\vec{B} + \mu_0 J + \frac{\partial \vec{E}}{\partial t}   \\
    & = \frac{\partial}{\partial t} \left(\vec{E} + ic\vec{B}\right) + \mu_0 J                  \\
\vec{\nabla} \wedge F
    & = \frac{\partial F}{\partial t} + \mu_0 J
\end{flalign}
